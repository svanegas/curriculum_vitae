%%%%%%%%%%%%%%%%%%%%%%%%%%%%%%%%%%%%%%%%%
% Medium Length Graduate Curriculum Vitae
% LaTeX Template
% Version 1.1 (9/12/12)
%
% This template has been downloaded from:
% http://www.LaTeXTemplates.com
%
% Original author:
% Rensselaer Polytechnic Institute (http://www.rpi.edu/dept/arc/training/latex/resumes/)
%
% Important note:
% This template requires the res.cls file to be in the same directory as the
% .tex file. The res.cls file provides the resume style used for structuring the
% document.
%
%%%%%%%%%%%%%%%%%%%%%%%%%%%%%%%%%%%%%%%%%

%---------------------------------------------------------------------------------------------------
%  PACKAGES AND OTHER DOCUMENT CONFIGURATIONS
%---------------------------------------------------------------------------------------------------

\documentclass[margin, 10pt]{res} % Use the res.cls style, the font size can be changed to 11pt or
                                  % 12pt here

\usepackage{helvet} % Default font is the helvetica postscript font
%\usepackage{newcent} % To change the default font to the new century schoolbook postscript font
                      % uncomment this line and comment the one above
\usepackage{hyperref}
\usepackage{graphicx}
\usepackage{amsmath}
\usepackage{amsfonts}
\usepackage[utf8]{inputenc}

\setlength{\textwidth}{5.1in} % Text width of the document

\begin{document}

%---------------------------------------------------------------------------------------------------
%  NAME AND ADDRESS SECTION
%---------------------------------------------------------------------------------------------------

\moveleft.5\hoffset\centerline{\large\bf Santiago Vanegas Gil} % Your name at the top

\moveleft\hoffset\vbox{\hrule width\resumewidth height 1pt}\smallskip % Horizontal line after name;
                                                                      % adjust line thickness by
                                                                      % changing the '1pt'

\moveleft.5\hoffset\centerline{\texttt{https://github.com/svanegas}} % Your address
\moveleft.5\hoffset\centerline{savanegasg@gmail.com}
\moveleft.5\hoffset\centerline{Birth: January 13th, 1995}
\moveleft.5\hoffset\centerline{Medellín, Antioquia, Colombia}

%---------------------------------------------------------------------------------------------------

\begin{resume}

%---------------------------------------------------------------------------------------------------
%  INTERESTS
%---------------------------------------------------------------------------------------------------

\section{INTERESTS}

Problem solving, coding, competitive programming contests, teaching, write and propose easy
programming problems, web development, design and develop mobile applications, aviation.

%---------------------------------------------------------------------------------------------------
%  SUMMARY
%---------------------------------------------------------------------------------------------------

\section{SUMMARY}

Back end, mobile and web developer with experience designing, developing and deploying applications
using Ruby, Ruby on Rails, Android, AngularJS, and Docker. \\
\newline
I have competed in Competitive Programming Contests, ranging from local practice contests to
South America finals. Furthermore, I have lectured the EAFIT's Competitive Programming Seminar
since January 2014. \\
\newline
I always concentrate on the quality of the products I work on, and strive to follow the best
practices and guidelines while developing. I always try to learn from my teammates, I like
to research and learn about tools, frameworks, new languages and algorithms that will help me to
get a better performance in the development process.

%---------------------------------------------------------------------------------------------------
%  EXPERIENCE
%---------------------------------------------------------------------------------------------------

\section{EXPERIENCE AND PROJECTS}

\textbf{Web Developer - Software Engineer at Talos Digital, Inc.} \\
\textit{January 2016 - Now (1 month)} \\
Developed applications in multiple technologies, such as \textbf{Ruby on Rails} in the back end and
\textbf{AngularJS}, Less and Sass in the front end, automating tasks using \textbf{GruntJS} and
\textbf{Gulp.js}, building applications using Docker basics.
\begin{itemize}
  \item \textbf{Flare App} \\
        \textit{January 2016 - Now (1 month)} \\
        Flare is a project which allows people to report incidents such as ambiental and medical
        issues to the zone they are. Zones are defined by admins of accounts. The project is
        developed using Laravel Framework and \textbf{AngularJS}. \\
        % \underline{Approximate contribution:} 85\% of the \textbf{AngularJS} application. \\
\end{itemize}

\textbf{Software Engineering Intern at Talos Digital, Inc.} \\
\textit{July 2015 - January 2016 (6 months)} \\
Full-stack developer. Working in micro-services back end in \textbf{Ruby on Rails}, using
\textbf{Grape} framework. Started learning and developing front end applications using
\textbf{AngularJS}, HTML and CSS.

\begin{itemize}
  \item \textbf{TaskFlex} \\
        \textit{August 2015 - December 2015 (5 months)} \\
        TaskFlex is an open-source project and reverse auction platform wich allows Owners to post
        and manage their jobs and Taskers to apply for different job offers. It allows to handle
        almost any service you want. The heavy logic about users and jobs is delegated to TDJobs and
        TDUser microservices. TaskFlex is built using \textbf{AngularJS} and Less.js. \\
        % \underline{Approximate contribution:} 85\% of the \textbf{AngularJS} application. \\
        %                                       60\% of the \textbf{Ruby on Rails} back end.

  \item \textbf{TDJobs} \\
        \textit{July 2015 - November 2015 (5 months)} \\
        Marketplace process abstraction API, built with \textbf{Ruby on Rails} and \textbf{Grape} on
        Docker, to handle jobs supply \& demand logic for applications with a business model like
        that of Task Rabbit or UpWork. \\
        % \underline{Approximate contribution:} 40\% of the \textbf{Ruby on Rails} back end. \\
\end{itemize}

\textbf{Teacher Assistant in Data Structures and Algorithms II at Universidad EAFIT} \\
\textit{January 2015 - June 2015 (6 months)} \\
Teaching data structures as maps, sets, trees; and common Computer Science algorithms for Graphs,
Dynamic Programming, Pattern Searching and Ad Hoc solutions.

\textbf{Mobile Devices Developer at IdeasLab} \\
\textit{July 2013 - June 2014 (1 year)} \\
Duing was a social application for mobile devices. My team and I developed the \textbf{Android}
native application for this platform, that required knowledge for development \textbf{graphic
components and obtaning data from Web Services}. \\
% \underline{Approximate contribution:} 85\% of the \textbf{Android} application. \\

\textbf{University projects at Universidad EAFIT} \\
\textit{January 2012 - Now} \\
\begin{itemize}
  \item \textbf{Nower}: is a mobile application which offers the users different promos for several
        stores, that is, you can look in a map for near stores in a range, see the promos they offer
        and take one. The application will give you a promo code which is used to redeem the promo
        in the store.\\
        The idea goes further than described above.\\
        \emph{We (my university team) started developing this idea in 2015-1, and we are going to
              continue developing it.}\\
        % \underline{Approximate contribution:} 90\% of the \textbf{Ruby on Rails} back end. \\
        %                                       20\% of the \textbf{Android} native application.
  \item Simple games developed in \textbf{Java}.
  \item A queue manager for \textbf{Android}, that is, request turns and being notified by mobile.
  \item A Web Image Hosting Application developed in HTML5, CSS3, JavaScript and PHP.
  \item CoffeeShop Cashier Application developed in Assembler.
  \item Several \textit{Coding Dojos} using Ionic Framework for \textbf{AngularJS}.
  \item Programming Labs developed in \textbf{C++} in order to understand specific Operating Systems
        concepts.
\end{itemize}
Check them out in my github! \texttt{https://github.com/svanegas}

%---------------------------------------------------------------------------------------------------
%  EDUCATION SECTION
%---------------------------------------------------------------------------------------------------

\section{EDUCATION}

\textbf{Universidad EAFIT} \\
\textit{January 2012 - (Expected) December 2016} \\
Backelor of Applied Science (BASc), Computer Science (Systems Engineering) \\
Grade: GPA: 4.52 out of 5.0
\begin{itemize}
  \item \textbf{Computer Science subjects:} Programming Fundamentals, Principles of Software
                Development, Programming Languages, Data Structures and Algorithms 1 and 2,
                Databases, Digital Electronics and Circuits, Digital Logic and Microcontrollers,
                Formal Languages and Compilers, Software Engineering,
                Technology Integration Project 1 and 2, Systemic Thinking, Computer Networks,
                Information Systems, Computer Graphics, Computer Architecture, Numerical Methods,
                Special Topics in Computer Networks, Operating Systems, Special Topics in Software
                Engineering.
  \item \textbf{Mathematics subjects:} Calculus, Predicate and Boolean Logic, Linear Algebra,
                                       Discrete Maths, Statistics, Quantitative Methods.
  \item \textbf{Organizations:} Competitive Programming Seminar, Mobile Devices Development Seminar.
\end{itemize}

%---------------------------------------------------------------------------------------------------
%  COMPUTER SKILLS SECTION
%---------------------------------------------------------------------------------------------------

\section{INFORMATICS KNOWLEDGE}
Proficient with algorithmic thinking and problem solving. \\
Most experience and preferred language: \textbf{\textit{C++}}. \\
Moderate experience with: \textbf{\textit{Java}} (Desktop and \textbf{Android}),
                          \textbf{\textit{Ruby}}, \textbf{\textit{Ruby on Rails}}. \\
Moderate experience with web programming and design using: \textbf{\textit{HTML5, AngularJS}}. \\
Automating tasks using: \textbf{\textit{GruntJS, Gulp.js}}. \\
Creating images and deploying containers using: \textbf{\textit{Docker}}. \\
Basic experience with database design and usage: \textbf{\textit{MySQL, PostgreSQL}}. \\
Learned basics for university projects: \textbf{\textit{Assembler}}. \\
Versioning and tools: \textbf{\textit{Git, Atom,} \LaTeX}.\\
Familiarity with operating Systems: \textbf{\textit{Linux, OS X}}. \\
Working on a team following: \textbf{\textit{Scrum}}.\\

%---------------------------------------------------------------------------------------------------
%  EXTRA-CURRICULAR ACTIVITIES SECTION
%---------------------------------------------------------------------------------------------------

\section{EXTRA-CURRICULAR / \\ VOLUNTEEER EXPERIENCE}
\textbf{Competitive Programming Seminar Student Coordinator at Universidad EAFIT} \\
Lecturer and student coordinator \\
\textit{January 2014 - Now (2 years 2 months)} \\
Coordinating students and teaching several Computer Science topics involved in Competitive
Programming Contests, such as \textbf{Dynamic Programming}, Greedy Algorithms, Pattern Matching;
and data structures like \textbf{Maps}, \textbf{Sets}, \textbf{Graphs}, Queues, Heaps. I manage our
virtual judge for internal contests. (We use BOCA Online Judge).

\textbf{Mobile Devices Development Seminar at Universidad EAFIT} \\
\textit{Member} \\
\textit{January 2015 - July 2015 (6 months)} \\
I joined this seminar because I really like developing for mobile devices, specially for
\textbf{Android}. In this seminar we created applications from good ideas that could help
students in their life. I've been through two projects in this group.
\begin{itemize}
  \item \textbf{Campus Móvil:} This application allowed EAFIT University students to easily locate
                               buildings and important places inside the campus.
  \item \textbf{Exams:} \emph{(Not completed yet)} The idea was to have a data base of previous
                        exams of a particular subject. User could upload a picture of his own exam
                        (share it), and search for exams by subject name, teacher name and other
                        parameters.
\end{itemize}


%---------------------------------------------------------------------------------------------------
%  CONTESTS
%---------------------------------------------------------------------------------------------------

\section{HONORS AND AWARDS}

\textbf{Participated at the XXVI Colombian Programming Contest} \\
ACIS REDIS \\
\textit{October, 2012} \\
This was my very first programming contest, my team could not advance to next round.

\textbf{10th place, XXVIII Colombian Programming Contest} \\
ACIS REDIS \\
\textit{September, 2014} \\
In my second national programming contest, we got the tenth place, advancing to South America-North
finals.

\textbf{20th place, South America - North Regional Programing Contest} \\
ACM-ICPC \\
\textit{November, 2014} \\
In my first South America-North finals we got the twentieth place, we didn't advance to World
Finals.

\textbf{10th place, XXIX Colombian Programming Contest} \\
ACIS REDIS \\
\textit{September, 2015} \\
This was my third national programming contest, we've got the tenth place, advancing to South
America-North finals again.

\textbf{11th place, South America - North Regional Programming Contest} \\
ACM-ICPC \\
\textit{November, 2015} \\
In my second regional final we could advance nine places, getting the eleventh place.
We needed to get the first, second or maybe third to advance to the World Finals.

\textbf{Participant of national competitive programming organizations and online judges} \\
Colombian Collegiate Programming League \& Red de Programación Competitiva, Google Code Jam, UVa,
Codeforces, SPOJ \\
\textit{January 2012 - Now} \\
Handle: {\sl svanegas}

%---------------------------------------------------------------------------------------------------
%  Languages
%---------------------------------------------------------------------------------------------------

\section{LANGUAGES}
Spanish: Native. \\
English: Advanced.

%---------------------------------------------------------------------------------------------------
\end{resume}
\end{document}
