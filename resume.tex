%%%%%%%%%%%%%%%%%%%%%%%%%%%%%%%%%%%%%%%%%
% Medium Length Graduate Curriculum Vitae
% LaTeX Template
% Version 1.1 (9/12/12)
%
% This template has been downloaded from:
% http://www.LaTeXTemplates.com
%
% Original author:
% Rensselaer Polytechnic Institute (http://www.rpi.edu/dept/arc/training/latex/resumes/)
%
% Important note:
% This template requires the res.cls file to be in the same directory as the
% .tex file. The res.cls file provides the resume style used for structuring the
% document.
%
%%%%%%%%%%%%%%%%%%%%%%%%%%%%%%%%%%%%%%%%%

%----------------------------------------------------------------------------------------
%  PACKAGES AND OTHER DOCUMENT CONFIGURATIONS
%----------------------------------------------------------------------------------------

\documentclass[margin, 10pt]{res} % Use the res.cls style, the font size can be changed to 11pt or
                                  % 12pt here

\usepackage{helvet} % Default font is the helvetica postscript font
%\usepackage{newcent} % To change the default font to the new century schoolbook postscript font
                      % uncomment this line and comment the one above
\usepackage{hyperref}
\usepackage{graphicx}
\usepackage{amsmath}
\usepackage{amsfonts}
\usepackage[utf8]{inputenc}

\setlength{\textwidth}{5.1in} % Text width of the document

\begin{document}

%----------------------------------------------------------------------------------------
%  NAME AND ADDRESS SECTION
%----------------------------------------------------------------------------------------

\moveleft.5\hoffset\centerline{\large\bf Santiago Vanegas Gil} % Your name at the top

\moveleft\hoffset\vbox{\hrule width\resumewidth height 1pt}\smallskip % Horizontal line after name;
                                                                      % adjust line thickness by
                                                                      % changing the '1pt'

\moveleft.5\hoffset\centerline{\texttt{https://github.com/svanegas}} % Your address
\moveleft.5\hoffset\centerline{savanegasg@gmail.com}
\moveleft.5\hoffset\centerline{Birth: January 13th, 1995}
\moveleft.5\hoffset\centerline{Medellín, Antioquia, Colombia}

%----------------------------------------------------------------------------------------

\begin{resume}

%----------------------------------------------------------------------------------------
%  INTERESTS
%----------------------------------------------------------------------------------------

\section{INTERESTS}

Problem solving, coding, competitive programming contests, teaching, writing easy programming
problems, web development, designing mobile applications, aviation.

%----------------------------------------------------------------------------------------
%  EDUCATION SECTION
%----------------------------------------------------------------------------------------

\section{EDUCATION}

Current Student of {\sl Systems Engineering}, Computer Science, EAFIT University, Medellín,
Antioquia, Colombia. \\
Starting date: January 2012 | Expected graduate: December 2016 \\
Current average grade: 4.45 out of 5.0
\begin{itemize}
  \item \textbf{Computer Science subjects:} Programming Fundamentals, Principles of Software
                Development, Programming Languages, Data Structures and Algorithms 1 and 2,
                Databases, Digital Electronics and Circuits, Digital Logic and Microcontrollers,
                Formal Languages and Compilers, Software Engineering,
                Technology Integration Project 1 and 2, Systemic Thinking, Computer Networks,
                Information Systems, Computer Graphics, Computer Architecture, Numerical Methods,
                Special Topics in Computer Networks, Operating Systems, Special Topics in Software
                Engineering.
  \item \textbf{Mathematics subjects:} Calculus, Predicate and Boolean Logic, Linear Algebra,
                                       Discrete Maths, Statistics, Quantitative Methods.
  \item \textbf{Organizations:} Competitive Programming Seminar, Mobile Devices Development Seminar.
\end{itemize}

%----------------------------------------------------------------------------------------
%  COMPUTER SKILLS SECTION
%----------------------------------------------------------------------------------------

\section{INFORMATICS KNOWLEDGE}
Proficient with algorithmic thinking and problem solving. \\
Most experience and preferred language: \textbf{\textit{C++}}. \\
Moderate experience with: \textbf{\textit{Java}} (Desktop and Android),
                          \textbf{\textit{Ruby}}, \textbf{\textit{Ruby on Rails}} \\
Moderate experience with web programming and design using: \textbf{\textit{HTML5, AngularJS}}. \\
Basic experience with database design and usage: \textbf{\textit{MySQL, PostgreSQL}}. \\
Learned basics for university projects: \textbf{\textit{Assembler}}. \\
Versioning and editors: \textbf{\textit{Git, Sublime Text, Atom}}.\\
Familiarity with operating Systems: \textbf{\textit{Linux, Windows, Mac OS}}.


%----------------------------------------------------------------------------------------
%  EXPERIENCE
%----------------------------------------------------------------------------------------

\section{EXPERIENCE AND PROJECTS}

\textbf{Part of mobile development team of a social application} \hfill (2013 - 2014)\\
\textit{{\sl Duing} (now named {\sl Reportt}) is a social application for mobile devices.}\\
My team developed the \textbf{Android} native application for {\sl Duing}, which required knowledge
for development of Android applications, focusing in several \textbf{graphic components and
obtaining data from Web Services}.\\
\underline{Approximate contribution:} 85\% of the Android application. \\
\emph{This application is in process to be published in Google Play Store}.\\
\textbf{University projects:} \hfill (2012 - Now)
\begin{itemize}
  \item \textbf{Nower}: is a mobile application which offers the users different promos for several stores,
          that is, you can look in a map for near stores in a range, see the promos they offer and take one.
          The application will give you a promo code, which is used to redeem the promo in the store.\\
          The idea goes further than described above.\\
          \emph{We (my university team) started developing this idea in 2015-1, and we are still developing it.}\\
          \underline{Approximate contribution:} 90\% of the \textbf{Ruby on Rails} back-end.\\
                                                                  20\% of the \textbf{Android} native application.
  \item Simple games developed in \textbf{Java}.
  \item A queue manager for \textbf{Android}, that is, request turns and being notified by mobile.
  \item A Web Image Hosting Application developed in \textbf{HTML5, CSS3, JavaScript and PHP}.
  \item CoffeeShop Cashier Application developed in \textbf{Assembler}.
  \item Several \textit{Coding Dojos} using native \textbf{Android} and \textbf{Ionic Framework} for
        \textbf{AngularJS}.
  \item Programming Labs developed in \textbf{C++} in order to understand specific Operating Systems
        concecpts.
\end{itemize}
Check them out in my github! \texttt{https://github.com/svanegas}

%----------------------------------------------------------------------------------------
%  EXTRA-CURRICULAR ACTIVITIES SECTION
%----------------------------------------------------------------------------------------

\section{EXTRA-CURRICULAR \\ ACTIVITIES}
\textbf{Competitive Programming Seminar - EAFIT University} \\
\textit{Lecturer assistant and student coordinator.} \\
Managing the intern online judge platform for the Seminar {\sl (We adapt BOCA Online Judge)},
where we use to solve problems. \texttt{http://boca.dis.eafit.edu.co/} \\
Writing and proposing simple problems that involve basic Data Structures and algorithms. \\
Lecturing and teaching several programming topics. \\

\textbf{Mobile Devices Development Seminar - EAFIT University} \\
\textit{Participant} \\
I joined to this seminar because I really like developing for mobile devices, specially for
\emph{Android}. In this seminar we create applications from good ideas that could help
students in their life. I've been through two projects in this group.

%----------------------------------------------------------------------------------------
%  CONTESTS
%----------------------------------------------------------------------------------------

\section{CONTESTS}
Participated in the Google Code Jam 2013, 2014 and 2015. \\
Participated at the XXVI Colombian Programming Contest ACIS REDIS, October 2012. \\ 
Participated at the XXVIII Colombian Programming Contest ACIS REDIS, September 2014. \\
Participant at the XXIX Colombian Programming Contest ACIS REDIS, September 2015. \\
Participated at the ACM-ICPC South American-North Regionals, November 2014. \\
Participant of two national competitive programming organizations: \emph{Colombian Collegiate
Programming League} and \emph{Red de Programación Competitiva}.\\
Solving different kind of algorithmic problems in several online judges, such as,
{\sl UVa, Codeforces, COJ, SPOJ.}\\ {\sl handle: svanegas}\\

%----------------------------------------------------------------------------------------
%  Languages
%----------------------------------------------------------------------------------------

\section{LANGUAGES}
Spanish: Native. \\
English: Advanced.

%----------------------------------------------------------------------------------------
\end{resume}
\end{document}
